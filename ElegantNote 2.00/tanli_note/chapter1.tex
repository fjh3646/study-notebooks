\section{绪论}
\subsection{弹性力学内容}
弹性力学,通常简称为弹性力学,又称为弹性理论,是固体力学的一个分支。弹性力学研究弹性体由于受外力作用,边界约束或温度改变等原因而发生的应力、应变和位移。
\subsection{几个基本概念}
\begin{enumerate}
	\item 外力是指其他物体对研究对象的作用力,可以分为体积力和表面力,两者也分别简称为体力和面力。
	\item 体力是分布在物体体积的力,如重力和惯性力。
	\item 应力就是物体内部单位面积上的内力。
	\item 应变是用来描述物体各部分线段长度和两线段夹角的改变。
	\item 位移就是位置的移动。
\end{enumerate}
\subsection{基本假定}
\begin{enumerate}
	\item 连续性假定
	\item 完全弹性假定
	\item 均匀性假定
	\item 各向同性假定
	\item 小变形假设
\end{enumerate}